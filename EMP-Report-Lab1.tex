\documentclass[12pt, a4paper]{report}

\usepackage[T2A]{fontenc}
\usepackage[utf8]{inputenc}
\usepackage[english,russian]{babel}
\usepackage[left = 1 cm, right = 1 cm, top = 2cm, bottom = 2 cm, bindingoffset = 0 cm]{geometry}
\usepackage{amsmath,amsfonts,amssymb,amsthm,mathtools}
\usepackage{wasysym}
\usepackage{float}

\usepackage{graphicx}
\graphicspath{{pictures/}}
\DeclareGraphicsExtensions{.pdf,.png,.jpg}

\usepackage{alltt}

\begin{document}
\begin{titlepage}
\newpage

\begin{center}
Министерство образования и науки Российской Федерации \\
Федеральное государственное автономное образовательное
учреждение высшего образования \\
Национальный исследовательский Нижегородский государственный
университет им. Н.И. Лобачевского \\
Институт информационных технологий, математики и механики \\
\end{center}

\vspace{12em}

\begin{center}
\textsc{\textbf{Отчёт по лабораторной работе}}\\
\textsc{\textbf{Уравнения математической физики}}
\end{center}

\vspace{14em}



\newbox{\lbox}
\savebox{\lbox}{\hbox{Петров Павел, Михайлова Екатерина}}
\newlength{\maxl}
\setlength{\maxl}{\wd\lbox}
\hfill\parbox{11cm}{
\hspace*{5cm}\hspace*{-5cm}Студенты:\hfill\hbox to\maxl{Петров Павел, Михайлова Екатерина\hfill}\\
\\
\hspace*{5cm}\hspace*{-5cm}Группа:\hfill\hbox to\maxl{381803-1}\\
}


\vspace{\fill}

\begin{center}
Нижний Новгород \\2021
\end{center}

\end{titlepage}       

\chapter{Задача Штурма-Лиувилля}

Задача Штурма-Лиувилля является простейшей задачей о поиске ортонормированной системы: найти те значения параметра $\lambda$, при которых существует нетривиальные решения задачи:
\[ X'' + \lambda X = 0 \]
\[ \alpha X(0) - \beta X'(0) = 0, \quad \alpha \geq 0, \beta \geq 0, \alpha + \beta > 0, \]
\[ \gamma X(l) + \delta X'(l) = 0, \quad \gamma \geq 0, \delta \geq 0, \gamma + \delta > 0, \]
а также найти эти решения. Такие значения параметра $\lambda$ называются \textbf{собственными значениями}, а соответствующие им нетривиальные решения - \textbf{собственными функциями}.


Из анализа ограничений на параметры были получены девять случаев значений параметров:
\begin{enumerate}
	\item $ \alpha > 0, \beta = 0, \gamma > 0, \delta = 0. $
	\item $ \alpha > 0, \beta = 0, \gamma = 0, \delta > 0. $
	\item $ \alpha = 0, \beta > 0, \gamma > 0, \delta = 0. $
	\item $ \alpha = 0, \beta > 0, \gamma = 0, \delta > 0. $
	\item $ \alpha > 0, \beta = 0, \gamma > 0, \delta > 0. $
	\item $ \alpha = 0, \beta > 0, \gamma > 0, \delta > 0. $
	\item $ \alpha > 0, \beta > 0, \gamma > 0, \delta = 0. $
	\item $ \alpha > 0, \beta > 0, \gamma = 0, \delta > 0. $
	\item $ \alpha > 0, \beta > 0, \gamma > 0, \delta > 0. $
\end{enumerate}


\section{Для всех возможных девяти случаев найти собственные числа и собственный функции задачи Штурма-Лиувилля (собственные функции отнормировать!)}


Рассмотрим разные значения $\lambda$ в задаче Штурма-Лиувилля.
\subsection{$\lambda = 0$}
В этом случае уравнение имеет вид:
\[ X'' = 0, \]
а его решение:
\[ X(x) = C_{1} x + C_{2}, \]
где $C_{1}$ и $C_{2}$ - произвольные постоянные. Подставим это решение в граничные условия и получим, что:
\[\alpha C_{2} - \beta C_{1} = 0, \quad \gamma (C_{1} l + C_{2}) + \delta C_{1} = 0. \]

\subsubsection{$ \alpha > 0, \beta = 0, \gamma > 0, \delta = 0.$}
В этом случае:
\[\alpha C_{2} = 0, \quad \gamma (C_{1} l + C_{2}) = 0, \]
\[ C_{2} = 0, \quad  \gamma C_{1} l = 0. \]
Так как по условию $\gamma$ и $l$ отличны от нуля, то равенство последнего уравнения нулю возможно только в одном случае, когда $C_{1} = 0$. Как итог:
\[ C_{1} = 0, \, C_{2} = 0, \, X(x) = 0. \]
Получили тривиальное решение, но оно нас не интересует.

\subsubsection{ $ \alpha > 0, \beta = 0, \gamma = 0, \delta > 0. $}
В этом случае:
\[\alpha C_{2} = 0, \quad \delta C_{1} = 0. \]
Исходя из условий, получаем:
\[ C_{1} = 0, \, C_{2} = 0, \, X(x) = 0. \]
Получили тривиальное решение, но оно нас не интересует.

\subsubsection{ $ \alpha = 0, \beta > 0, \gamma > 0, \delta = 0. $}
В этом случае:
\[ - \beta C_{1} = 0, \quad \gamma (C_{1} l + C_{2}) = 0, \]
\[ C_{1} = 0, \quad \gamma C_{2} = 0. \]
Так как по условию $\gamma$ отлична от нуля, то равенство последнего уравнения нулю возможно только в одном случае, когда $C_{2} = 0$. Как итог:
\[ C_{1} = 0, \, C_{2} = 0, \, X(x) = 0. \]
Получили тривиальное решение, но оно нас не интересует.

\subsubsection{ $ \alpha = 0, \beta > 0, \gamma = 0, \delta > 0. $}
В этом случае:
\[-\beta C_{1} = 0, \quad \delta C_{1} = 0. \]
Исходя из условий, получаем, что $C_{1} = 0$, а на $C_{2}$ ограничений нет. Значит, чтобы получить нетривиальное решение, надо взять $C_{2}$ произвольной константой, отличной от нуля. Итог:
\[ C_{1} = 0, \, C_{2} = const \neq 0, \, X(x) = C_{2}. \]
Получили нетривиальное решение.

\subsubsection{ $ \alpha > 0, \beta = 0, \gamma > 0, \delta > 0. $}
В этом случае:
\[ \alpha C_{2} = 0, \quad \gamma (C_{1} l + C_{2}) + \delta C_{1} = 0, \]
\[ C_{2} = 0, \quad (\gamma l + \delta) C_{1} = 0. \]
Так как по условию $\gamma$, $l$ и $\delta$ строго больше нуля, то равенство последнего уравнения нулю возможно только в одном случае, когда $C_{1} = 0$. Как итог:
\[ C_{1} = 0, \, C_{2} = 0, \, X(x) = 0. \]
Получили тривиальное решение, но оно нас не интересует.

\subsubsection{ $ \alpha = 0, \beta > 0, \gamma > 0, \delta > 0. $}
В этом случае:
\[ - \beta C_{1} = 0, \quad \gamma (C_{1} l + C_{2}) + \delta C_{1} = 0, \]
\[ C_{1} = 0, \quad \gamma C_{2} = 0. \]
Так как по условию $\gamma$ отлична от нуля, то равенство последнего уравнения нулю возможно только в одном случае, когда $C_{2} = 0$. Как итог:
\[ C_{1} = 0, \, C_{2} = 0, \, X(x) = 0. \]
Получили тривиальное решение, но оно нас не интересует.

\subsubsection{ $ \alpha > 0, \beta > 0, \gamma > 0, \delta = 0. $}
В этом случае:
\[ \alpha C_{2} - \beta C_{1} = 0, \quad \gamma (C_{1} l + C_{2}) = 0, \]
\[ C_{2} = \frac{\beta}{\alpha} C_{1}, \quad (\gamma l + \frac{\gamma \beta}{\alpha}) C_{1} = 0. \]
Так как по условию все представленные параметры строго больше нуля, то равенство последнего уравнения нулю возможно только в одном случае, когда $C_{1} = 0$, следовательно и $C_{2} = 0$. Как итог:
\[ C_{1} = 0, \, C_{2} = 0, \, X(x) = 0. \]
Получили тривиальное решение, но оно нас не интересует.

\subsubsection{ $ \alpha > 0, \beta > 0, \gamma = 0, \delta > 0. $}
В этом случае:
\[ \alpha C_{2} - \beta C_{1} = 0, \quad \delta C_{1} = 0, \]
\[ \alpha C_{2} = 0, \quad C_{1} = 0. \]
Так как по условию $\alpha$ строго больше нуля, то равенство первого уравнения нулю возможно только в одном случае, когда $C_{2} = 0$. Как итог:
\[ C_{1} = 0, \, C_{2} = 0, \, X(x) = 0. \]
Получили тривиальное решение, но оно нас не интересует.

\subsubsection{ $ \alpha > 0, \beta > 0, \gamma > 0, \delta > 0. $}
В этом случае:
\[ \alpha C_{2} - \beta C_{1} = 0, \quad \gamma (C_{1} l + C_{2}) + \delta C_{1} = 0, \]
\[ C_{2} = \frac{\beta}{\alpha} C_{1}, \quad (\gamma l + \delta + \frac{\gamma \beta}{\alpha}) C_{1} = 0. \]
Так как по условию все представленные параметры строго больше нуля, то равенство последнего уравнения нулю возможно только в одном случае, когда $C_{1} = 0$, следовательно и $C_{2} = 0$. Как итог:
\[ C_{1} = 0, \, C_{2} = 0, \, X(x) = 0. \]
Получили тривиальное решение, но оно нас не интересует.

\subsection{$\lambda < 0$}
В этом случае решение уравнения имеет вид:
\[ X(x) = C_{1} e^{-\sqrt{-\lambda}x} + C_{2} e^{\sqrt{-\lambda}x}. \]
Заменим $\sqrt{-\lambda}$ на $\mu$ и подставим решение в граничные условия. Получим:
\begin{displaymath}
	\begin{cases}
		\alpha (C_{1} + C_{2}) - \beta \mu (C_{2} - C_{1}) = 0, \\
		\gamma (C_{1} e^{-\mu l} + C_{2} e^{\mu l}) + \delta \mu (C_{2} e^{\mu l} - C_{1} e^{-\mu l}) = 0
	\end{cases}
\end{displaymath}

\subsubsection{ $ \alpha > 0, \beta = 0, \gamma > 0, \delta = 0. $}
В этом случае:
\begin{displaymath}
	\begin{cases}
		\alpha (C_{1} + C_{2}) = 0, \\
		\gamma (C_{1} e^{-\mu l} + C_{2} e^{\mu l}) = 0
	\end{cases}
\end{displaymath}

\begin{displaymath}
	\begin{cases}
		C_{1} = - C_{2}, \\
		\gamma C_{1}(e^{-\mu l} - e^{\mu l}) = 0
	\end{cases}
\end{displaymath}
Интересуемся нетривиальными решениями, поэтому приравняем $e^{-\mu l} - e^{\mu l}$ к нулю. Но здесь равенство нулю возможно только в случае равенства нулю $\mu l$, что невозможно, исходя из условий. Поэтому возможно только $C_{1} = 0$ и, следовательно, $C_{2} = 0$. Таким образом получаем тривиальное решение, что нам не подходит.

\[ C_{1} = 0, \, C_{2} = 0, \, X(x) = 0. \]

\subsubsection{ $ \alpha > 0, \beta = 0, \gamma = 0, \delta > 0. $}
В этом случае:
\begin{displaymath}
	\begin{cases}
		\alpha (C_{1} + C_{2}) = 0, \\
		\delta \mu (C_{2} e^{\mu l} - C_{1} e^{-\mu l}) = 0
	\end{cases}
\end{displaymath}

\begin{displaymath}
	\begin{cases}
		C_{1} = - C_{2}, \\
		-\delta \mu C_{1} (e^{\mu l} + e^{-\mu l}) = 0
	\end{cases}
\end{displaymath}
Второе уравнения равно нулю только в одном случае, когда $C_{1} = 0$, так как по условию параметры строго больше нуля, а $e^{\mu l} + e^{-\mu l}$ никогда не может быть равно нулю, так как представляет собой сумму положительных функций. Следовательно, $C_{2} = 0$, и получаем тривиальное решение, что нам не подходит.

\[ C_{1} = 0, \, C_{2} = 0, \, X(x) = 0. \]

\subsubsection{ $ \alpha = 0, \beta > 0, \gamma > 0, \delta = 0. $}
В этом случае:
\begin{displaymath}
	\begin{cases}
		- \beta \mu (C_{2} - C_{1}) = 0, \\
		\gamma (C_{1} e^{-\mu l} + C_{2} e^{\mu l}) = 0
	\end{cases}
\end{displaymath}

\begin{displaymath}
	\begin{cases}
		C_{2} = C_{1}, \\
		\gamma C_{1} (e^{-\mu l} + e^{\mu l}) = 0
	\end{cases}
\end{displaymath}
Второе уравнения равно нулю только в одном случае, когда $C_{1} = 0$, так как по условию параметр строго больше нуля, а $e^{\mu l} + e^{-\mu l}$ никогда не может быть равно нулю, так как представляет собой сумму положительных функций. Следовательно, $C_{2} = 0$, и получаем тривиальное решение, что нам не подходит.

\[ C_{1} = 0, \, C_{2} = 0, \, X(x) = 0. \]

\subsubsection{ $ \alpha = 0, \beta > 0, \gamma = 0, \delta > 0. $}
В этом случае:
\begin{displaymath}
	\begin{cases}
		- \beta \mu (C_{2} - C_{1}) = 0, \\
		\delta \mu (C_{2} e^{\mu l} - C_{1} e^{-\mu l}) = 0
	\end{cases}
\end{displaymath}

\begin{displaymath}
	\begin{cases}
		C_{2} = C_{1}, \\
		\delta \mu C_{1} (e^{\mu l} - e^{-\mu l}) = 0
	\end{cases}
\end{displaymath}
Интересуемся нетривиальными решениями, поэтому приравняем $e^{\mu l} - e^{-\mu l}$ к нулю. Но здесь равенство нулю возможно только в случае равенства нулю $\mu l$, что невозможно, исходя из условий. Поэтому возможно только $C_{1} = 0$ и, следовательно, $C_{2} = 0$. Таким образом получаем тривиальное решение, что нам не подходит.

\[ C_{1} = 0, \, C_{2} = 0, \, X(x) = 0. \]

\subsubsection{ $ \alpha > 0, \beta = 0, \gamma > 0, \delta > 0. $}
В этом случае:
\begin{displaymath}
	\begin{cases}
		\alpha (C_{1} + C_{2}) = 0, \\
		\gamma (C_{1} e^{-\mu l} + C_{2} e^{\mu l}) + \delta \mu (C_{2} e^{\mu l} - C_{1} e^{-\mu l}) = 0
	\end{cases}
\end{displaymath}

\begin{displaymath}
	\begin{cases}
		C_{1} = -C_{2}, \\
		C_{1} (\gamma (e^{-\mu l} - e^{\mu l}) - \delta \mu (e^{\mu l} + e^{-\mu l})) = 0
	\end{cases}
\end{displaymath}
Интересуемся нетривиальными решениями, поэтому во втором уравнении занулим скобку:
\[ \gamma (e^{-\mu l} - e^{\mu l}) - \delta \mu (e^{\mu l} + e^{-\mu l}) = 0, \]
затем умножим на $ e^{\mu l}$ и соберём слагаемые с экспонентами и без:
\[ (\gamma - \delta \mu) -  (\gamma + \delta \mu) e^{2 \mu l} = 0. \]
Получаем уравнение, которое надо решить относительно $\mu$:
\[ e^{2 \mu l} = \frac{\gamma - \delta \mu}{\gamma + \delta \mu}. \]
Слева в уравнении, очевидно, стоит монотонно возрастающая функция. Справа имеем гиперболу с вертикальной асимптотой $\mu = -\frac{\gamma}{\delta} < 0$. Но $\mu = \sqrt{-\lambda}$, поэтому рассматриваем только положительные $\mu$. Производная функции справа по $\mu$ равна: $\frac{-2\delta \gamma}{(\delta \mu + \gamma)^2} < 0$, значит гипербола убывает на всей своей области определения. При $\mu = 0$ функции совпадают, затем расходятся, поэтому при положительных $\mu$ обращение скобки в нуль невозможно. Значит возможно только $C_{1} = 0$ и, следовательно, $C_{2} = 0$, и решение тривиально.

\[ C_{1} = 0, \, C_{2} = 0, \, X(x) = 0. \]

\subsubsection{ $ \alpha = 0, \beta > 0, \gamma > 0, \delta > 0. $}
В этом случае: 
\begin{displaymath}
	\begin{cases}
		- \beta \mu (C_{2} - C_{1}) = 0, \\
		\gamma (C_{1} e^{-\mu l} + C_{2} e^{\mu l}) + \delta \mu (C_{2} e^{\mu l} - C_{1} e^{-\mu l}) = 0
	\end{cases}
\end{displaymath}

\begin{displaymath}
	\begin{cases}
		C_{2} = C_{1}, \\
		C_{1} (\gamma (e^{-\mu l} + e^{\mu l}) + \delta \mu (e^{\mu l} - e^{-\mu l})) = 0
	\end{cases}
\end{displaymath}
Интересуемся нетривиальными решениями, поэтому во втором уравнении занулим скобку:
\[ \gamma (e^{-\mu l} + e^{\mu l}) + \delta \mu (e^{\mu l} - e^{-\mu l}) = 0, \]
затем умножим на $ e^{\mu l}$ и соберём слагаемые с экспонентами и без:
\[ (\gamma - \delta \mu) + (\gamma + \delta \mu) e^{2 \mu l} = 0. \]
Получаем уравнение, которое надо решить относительно $\mu$:
\[ e^{2 \mu l} = \frac{\delta \mu - \gamma}{\delta \mu + \gamma}. \]
Слева в уравнении, очевидно, стоит монотонно возрастающая функция. Справа имеем гиперболу с вертикальной асимптотой $\mu = -\frac{\gamma}{\delta} < 0$. Но $\mu = \sqrt{-\lambda}$, поэтому рассматриваем только положительные $\mu$. Производная функции справа по $\mu$ равна: $\frac{2\delta \gamma}{(\delta \mu + \gamma)^2} < 0$, значит гипербола возрастает на всей своей области определения. При $\mu = 0$ функция слева равна 1, справа равна -1, при стремлении $\mu$ к бесконечности, гипербола стремится к 1, пока экспонента в это время уходит на бесконечность, поэтому нет такого значения $\mu$, при котором скобка обращалась бы в нуль. Значит возможно только $C_{1} = 0$ и, следовательно, $C_{2} = 0$, и решение тривиально.

\[ C_{1} = 0, \, C_{2} = 0, \, X(x) = 0. \]

\subsubsection{ $ \alpha > 0, \beta > 0, \gamma > 0, \delta = 0. $}

В этом случае:

\begin{equation*}
 \begin{cases}
 C_1(\alpha -\beta)+\sqrt{-\lambda} C_2(\alpha+\beta) =0, 
   \\
C_1\sqrt{-\lambda} e^{\sqrt{-\lambda}l}-C_2\sqrt{-\lambda} e^{\sqrt{-\lambda}l}=0
   \\  
 \end{cases}
\end{equation*}
Пусть $C_1$ и $C_2$ не обращаются одновременно в 0:
Выразим выражение $\frac{C_1}{C_2}$ через первое и второе уравнение в системе:

\[\frac{C_1}{C_2} = e^{-2\sqrt{-\lambda}l}=-\frac{\sqrt{-\lambda}(\alpha+\beta)}{\beta-\alpha}\]
Это уравнение не имеет решений.
Из второго уравнения следует, что равняться нулю только одна из констант не может, поэтому наша задача имеет только тривиальное решение  $X_n(x)=0$.

\subsubsection{ $ \alpha > 0, \beta > 0, \gamma = 0, \delta > 0. $}
В этом случае:
\begin{displaymath}
	\begin{cases}
		\alpha (C_{1} + C_{2}) - \beta \mu (C_{2} - C_{1}) = 0, \\
		\delta \mu (C_{2} e^{\mu l} - C_{1} e^{-\mu l}) = 0
	\end{cases}
\end{displaymath}
Выразим $C_{2}$ через $C_{1}$ в первом уравнении:
\begin{displaymath}
	\begin{cases}
		C_{2} = -\frac{\beta \mu + \alpha}{\beta \mu - \alpha} C_{1}, \\
		- \delta \mu C_{1} (\frac{\beta \mu + \alpha}{\beta \mu - \alpha} e^{\mu l} + e^{-\mu l}) = 0
	\end{cases}
\end{displaymath}
Домножим последнее уравнение на $e^{\mu l}$ и $\frac{\beta \mu - \alpha}{\beta \mu + \alpha}$ и занулим скобку:
\[ e^{2\mu l} = -\frac{\beta \mu - \alpha}{\beta \mu + \alpha} \]
Слева в уравнении, очевидно, стоит монотонно возрастающая функция. Справа имеем гиперболу с вертикальной асимптотой $\mu = -\frac{\alpha}{\beta} < 0$. Но $\mu = \sqrt{-\lambda}$, поэтому рассматриваем только положительные $\mu$. Производная функции справа по $\mu$ равна: $\frac{-2\alpha \beta}{(\beta \mu + \alpha)^2} < 0$, значит гипербола убывает на всей своей области определения. При $\mu = 0$ функции совпадают, затем расходятся, поэтому при положительных $\mu$ обращение скобки в нуль невозможно. Значит возможно только $C_{1} = 0$ и, следовательно, $C_{2} = 0$, и решение тривиально.

\[ C_{1} = 0, \, C_{2} = 0, \, X(x) = 0. \]

\subsubsection{ $ \alpha > 0, \beta > 0, \gamma > 0, \delta > 0. $}
В этом случае:
\begin{displaymath}
	\begin{cases}
		\alpha (C_{1} + C_{2}) - \beta \mu (C_{2} - C_{1}) = 0, \\
		\gamma (C_{1} e^{-\mu l} + C_{2} e^{\mu l}) + \delta \mu (C_{2} e^{\mu l} - C_{1} e^{-\mu l}) = 0
	\end{cases}
\end{displaymath}
Выразим $C_{2}$ через $C_{1}$ в первом уравнении:
\begin{displaymath}
	\begin{cases}
		C_{2} = -\frac{\beta \mu + \alpha}{\beta \mu - \alpha} C_{1}, \\
		\gamma (C_{1} e^{-\mu l} -\frac{\beta \mu + \alpha}{\beta \mu - \alpha} C_{1} e^{\mu l}) + \delta \mu (-\frac{\beta \mu + \alpha}{\beta \mu - \alpha} C_{1} e^{\mu l} - C_{1} e^{-\mu l}) = 0
	\end{cases}
\end{displaymath}
Умножим последнее уравнение на $e^{\mu l} и вынесем C_{1}$:
\begin{displaymath}
	\begin{cases}
		C_{2} = -\frac{\beta \mu + \alpha}{\beta \mu - \alpha} C_{1}, \\
		\gamma C_{1}(1 -\frac{\beta \mu + \alpha}{\beta \mu - \alpha} e^{2 \mu l}) + \delta \mu C_{1} (-\frac{\beta \mu + \alpha}{\beta \mu - \alpha} e^{2\mu l} - 1) = 0
	\end{cases}
\end{displaymath}
Преобразуем второй уравнение:
\[ C_{1} ( -\frac{\beta \mu + \alpha}{\beta \mu - \alpha} (\gamma + \delta \mu ) e^{2 \mu l} - (\delta \mu - \gamma)) = 0 \]
Приравняем скобку к нулю и получим уравнение, которое нужно решить относительно $\mu$:
\[ e^{2\mu l} = - \frac{\beta \mu - \alpha}{\beta \mu + \alpha} \frac{\delta \mu - \gamma}{\delta \mu + \gamma}\]
При $\mu = 0$ экспонента равна 1, а функция справа -1, на бесконечности экспонента устремляется к бесконечности, монотонно возрастая, а функция справа стремится к -1, причём скорость роста экспоненты выше, чем скорость функции справа, поэтому пересечения графиков быть не может. Следовательно, скобка не может обратиться к нуль, а значит равенство нулю второго уравнения системы возможно только лишь в случае $C_{1} = 0$. Из этого следует, что $C_{2} = 0$ и решение тривиально.

\[ C_{1} = 0, \, C_{2} = 0, \, X(x) = 0. \]

\subsection{$\lambda > 0$}

В этом случае решение уравнения имеет вид:
\[ X(x) = C_{1} \cos \sqrt{\lambda} x + C_{2} \sin \sqrt{\lambda}x. \]
Заменим $\sqrt{\lambda}$ на $\mu$ и подставим решение в граничные условия. Получим:
\begin{displaymath}
	\begin{cases}
		\alpha C_{1} - \beta \mu C_{2} = 0, \\
		\gamma (C_{1} \cos \mu l + C_{2} \sin \mu l) + \delta \mu ( -C_{1} \sin \mu l + C_{2} \cos \mu l) = 0
	\end{cases}
\end{displaymath}

\subsubsection{ $ \alpha > 0, \beta = 0, \gamma > 0, \delta = 0. $}

В этом случае:

\begin{equation*}
 \begin{cases}
\alpha C_2=0, 
   \\
\gamma C_1 sin(\sqrt{\lambda}l )=0
   \\  
 \end{cases}
\end{equation*}

\begin{equation*}
 \begin{cases}
C_2=0, 
   \\
C_1 sin(\sqrt{\lambda}l )=0
   \\  
 \end{cases}
\end{equation*}
Интересуемся нетривиальными решениями, $C_1$ не может быть равно 0, поэтому приравняем $sin(\sqrt{\lambda}l) $ к нулю: 
\[sin(\sqrt{\lambda}l) =0,\]
\[\sqrt{\lambda}l = \pi n, n \in N,\]
\[\lambda_n = \frac{\pi^2 n^2}{l^2}, n \in N.\]
Найдем собственные функции:
\[X_n(x)=C_{k}sin(\frac{\pi n}{l}x),  n \in N,\]
Отнормируем собственные функции:
\[1=(C_{k}sin(\frac{\pi n}{l}x), C_{k}sin(\frac{\pi n}{l}x))=C_{k}^2 \int_{0}^{l} \sin(\frac{\pi n}{l}x)dx=\frac{C_{k}^2l}{2}\]
Отсюда:
\[C_{k}=\sqrt{\frac{2}{l}}\]

\subsubsection{ $ \alpha > 0, \beta = 0, \gamma = 0, \delta > 0. $}

В этом случае:
\begin{equation*}
 \begin{cases}
\alpha C_2=0, 
   \\
\delta C_1 cos (\sqrt{\lambda}l)=0
   \\  
 \end{cases}
\end{equation*}

\begin{equation*}
 \begin{cases}
C_2=0, 
   \\
C_1 cos (\sqrt{\lambda}l )=0
   \\  
 \end{cases}
\end{equation*}
Интересуемся нетривиальными решениями, поэтому приравняем $cos(\sqrt{\lambda}l ) $к нулю

\[cos(\sqrt{\lambda}l) =0,\]

\[\sqrt{\lambda}l = \frac{\pi}{2} +\pi n, n \in N,\]

Отсюда 

\[\lambda_n = \frac{\pi^2(n-\frac{1}{2})^2}{l^2}, n \in N\]

Найдем собственные функции:

\[X_n(x)=C_{K}sin(\frac{\pi(n-\frac{1}{2})}{l}x), n \in N\]
Отнормируем собственные функции:
\[1=(C_{k}sin(\frac{\pi(n-\frac{1}{2})}{l}x)), C_{k}sin(\frac{\pi(n-\frac{1}{2})}{l}x))=C_{k}^2 \int_{0}^{l} \sin^2(\frac{\pi(n-\frac{1}{2})}{l}x)dx=\frac{C_{k}^2l}{2}\]
Отсюда:
\[C_{k}=\sqrt{\frac{2}{l}}\]

\subsubsection{ $ \alpha = 0, \beta > 0, \gamma > 0, \delta = 0. $}

В этом случае:

\begin{equation*}
 \begin{cases}
\beta C_1 \sqrt{\lambda}=0, 
   \\
\gamma C_2 cos (\sqrt{\lambda}l )=0
   \\  
 \end{cases}
\end{equation*}

\begin{equation*}
 \begin{cases}
C_1=0, 
   \\
C_2 cos (\sqrt{\lambda}l )=0
   \\  
 \end{cases}
\end{equation*}
Интересуемся нетривиальными решениями, поэтому приравняем $cos(\sqrt{\lambda}l ) $к нулю

\[cos(\sqrt{\lambda}l) =0\]

\[\sqrt{\lambda}l = \frac{\pi}{2} +\pi n, n \in N\]

Отсюда 

\[\lambda_n = \frac{\pi^2(n-\frac{1}{2})^2}{l^2}, n \in N\]

Найдем собственные функции:

\[X_n(x)=C_{k}cos(\frac{\pi(n-\frac{1}{2})}{l}x), n \in N\]
Отнормируем собственные функции:
\[1=(C_{k}cos(\frac{\pi(n-\frac{1}{2})}{l}x),C_{k}cos(\frac{\pi(n-\frac{1}{2})}{l}x))=C_{k}^2\int_{0}^{l} \cos^2(\frac{\pi(n-\frac{1}{2})}{l}x)dx=\frac{C_{k}^2l}{2}\]
Отсюда:
\[C_{k}=\sqrt{\frac{2}{l}}\]

\subsubsection{ $ \alpha = 0, \beta > 0, \gamma = 0, \delta > 0. $}

В этом случае:

\begin{equation*}
 \begin{cases}
\beta C_1 \sqrt{\lambda}=0, 
   \\
\delta \sqrt{\lambda} C_2 sin (\sqrt{\lambda}l )=0
   \\  
 \end{cases}
\end{equation*}

\begin{equation*}
 \begin{cases}
C_1=0, 
   \\
C_2 sin (\sqrt{\lambda}e )=0
   \\  
 \end{cases}
\end{equation*}
Интересуемся нетривиальными решениями, поэтому приравняем $sin(\sqrt{\lambda}l ) $к нулю

\[sin(\sqrt{\lambda}l ) =0\]

\[\sqrt{\lambda}l = \pi n, n \in N\]

Отсюда 

\[\lambda_n = \frac{\pi^2n^2}{l^2}, n \in N\]

Найдем собственные функции:

\[X_n(x)=C_{k}cos(\frac{\pi n}{l}x), n \in N\]
Отнормируем собственные функции:
\[1=(C_{k}cos(\frac{\pi n}{l}x),C_{k}cos(\frac{\pi n}{l}x))=C_{k}^2\int_{0}^{l} \cos^2(\frac{\pi n}{l}x)x)dx=\frac{C_{k}^2l}{2}\]
Отсюда:
\[C_{k}=\sqrt{\frac{2}{l}}\]


\subsubsection{ $ \alpha > 0, \beta = 0, \gamma > 0, \delta > 0. $}

В этом случае:

\begin{equation*}
 \begin{cases}
\alpha C_2 =0, 
   \\
C_1(\gamma sin(\sqrt{\lambda}l)+\delta \sqrt{\lambda} cos(\sqrt{\lambda}l))=0
   \\  
 \end{cases}
\end{equation*}

Интересуемся нетривиальными решениями, поэтому приравняем $\gamma sin(\sqrt{\lambda}l)+\delta \sqrt{\lambda} cos(\sqrt{\lambda}l)$ к нулю

\begin{equation*}
 \begin{cases}
C_2=0, 
   \\
\gamma sin(\sqrt{\lambda}l)=-\delta \sqrt{\lambda} cos(\sqrt{\lambda}l)
   \\  
 \end{cases}
\end{equation*}

Так как $cos(\sqrt{\lambda}l)$ и $sin(\sqrt{\lambda}l)$ не могут одновременно равняться 0:

\[\lambda = \frac{\gamma^2}{\delta^2} tg^2(\sqrt{\lambda} l)\]

это уравнение имеет бесконечно много решений.

Найдем собственные функции:

\[X_n(x)=sin(\sqrt{\lambda_n}x), n \in N\]

\subsubsection{ $ \alpha = 0, \beta > 0, \gamma > 0, \delta > 0. $}
В этом случае:
\begin{displaymath}
	\begin{cases}
		C_{2} = 0, \\
		C_{1} (\gamma \cos \mu l - \delta \mu \sin \mu l) = 0
	\end{cases}
\end{displaymath}
Ищем нетривиальные решения, поэтому занулим во втором уравнении скобку и сделаем некоторое преобразование:
\[ \sqrt{\gamma ^2 + (\delta \mu)^2 }(\frac{\gamma}{\sqrt{\gamma ^2 + (\delta \mu)^2 }} \cos \mu l - \frac{\delta \mu}{\sqrt{\gamma ^2 + (\delta \mu)^2 }}\sin \mu l) = 0 \]
Обозначим $\tg \Omega = \frac{\delta \mu}{\gamma}$ и свернём получившееся выражение в скобке по формуле косинуса суммы:
\[ \sqrt{\gamma ^2 + (\delta \mu)^2 } cos(\mu l + \arctg  \frac{\delta \mu}{\gamma}) = 0 \]
Из условий на параметры и на собственное число получим, что:
\[ \mu l + \arctg  \frac{\delta \mu}{\gamma} = \frac{\pi}{2} + \pi k, \, k \in \mathbb{N}. \]
Слева получили, очевидно, монотонно возрастающую функцию, справа уравнение горизонтальной прямой. Понятно, что, в силу свойств функций, обязательно найдётся при каждом $k$ такое $\mu ^{*}_{k}$, которое будет удовлетворять полученному уравнению. Значит $C_{1} \ne 0$ и получено нетривиальное решение (или собственная функция), соответствующая собственному числу $\mu ^{*}_{k}$.

\[ C_{1} = const \ne 0, \, C_{2} = 0, \, X(x) = C_{1} \cos \sqrt{\lambda_{k}} x, \, \sqrt{\lambda_{k}} \in \{\sqrt{\lambda} \, | \sqrt{\lambda} l + \arctg  \frac{\delta \sqrt{\lambda}}{\gamma} = \frac{\pi}{2} + \pi k, \, k \in \mathbb{N}\}. \]
Отнормируем собственную функцию:
\[ 1 = (C_{k} \cos \sqrt{\lambda_{k}} x, C_{k} \cos \sqrt{\lambda_{k}} x) = C_{k}^2 \int_{0}^{l}  \cos^2 \sqrt{\lambda_{k}} xdx = C_{k}^2 (\frac{x}{2} + \frac{1}{2\lambda_{k}} \sin 2\lambda_{k} x)_0^l =   C_{k}^2 (\frac{l}{2} + \frac{1}{2\lambda_{k}} \sin 2\lambda_{k} l) \]
Отсюда:
\[ C_{k} = \sqrt{\frac{1}{\frac{l}{2} + \frac{1}{2\lambda_{k}} \sin 2\lambda_{k} l}}. \]

\subsubsection{ $ \alpha > 0, \beta > 0, \gamma > 0, \delta = 0. $}
В этом случае:
\begin{displaymath}
	\begin{cases}
		\alpha C_{1} - \beta \mu C_{2} = 0, \\
		\gamma (C_{1} \cos \mu l + C_{2} \sin \mu l) = 0
	\end{cases}
\end{displaymath}
Выразим $C_{2}$ через $C_{1}$ и подставим во второе уравнение:
\begin{displaymath}
	\begin{cases}
		C_{2} = \frac{\alpha}{\beta \mu} C_{1}, \\
		C_{1} (\cos \mu l + \frac{\alpha}{\beta \mu} \sin \mu l) = 0
	\end{cases}
\end{displaymath}
Преобразуем:
\begin{displaymath}
	\begin{cases}
		C_{2} = \frac{\alpha}{\beta \mu} C_{1}, \\
		C_{1} (\beta \mu \cos \mu l + \alpha \sin \mu l) = 0
	\end{cases}
\end{displaymath}
Интересуемся нетривиальными решениями, поэтому во втором уравнении занулим скобку и проведём некоторое преобразование:
\[ \sqrt{(\beta \mu)^2 + \alpha^2} (\frac{\beta \mu}{\sqrt{(\beta \mu)^2 + \alpha^2}} \cos \mu l + \frac{\alpha}{\sqrt{(\beta \mu)^2 + \alpha^2}} \sin \mu l) = 0 \]
В силу ограничений на параметры и собственное число равенство нулю возможно только в том случае, когда выражение, получившееся в скобке равно нулю. Свернём его по формуле синуса суммы, при условии что $\tg \Omega = \frac{\beta \mu}{\alpha}$:
\[ \sin (\mu l + \arctg \frac{\beta \mu}{\alpha}) = 0 \]
Нужно решить следующее уравнение относительно $\mu$:
\[ \mu l + \arctg \frac{\beta \mu}{\alpha} = \pi k, \, k \in \mathbb{N}. \]
Очевидно, что слева стоит монотонно возрастающая функция, справа функция, график которой является горизонтальной прямой, поэтому для любого $k$ найдётся $\mu ^{*}_{k}$, которое будет удовлетворять полученному уравнению. Значит $C_{1} \ne 0$ и получено нетривиальное решение (или собственная функция), соответствующая собственному числу $\mu ^{*}_{k}$.

\[ C_{1} = const \ne 0, \, C_{2} = \frac{\alpha}{\beta \mu_{k}} C_{1}, \, X(x) = C_{k} (\cos \sqrt{\lambda_{k}} x + \frac{\alpha}{\beta \mu} \sin \sqrt{\lambda_{k}} x), \]
\[ \sqrt{\lambda_{k}} \in \{ \sqrt{\lambda} \, | \, \sqrt{\lambda} l + \arctg \frac{\beta \sqrt{\lambda}}{\alpha} = \pi k, \, k \in \mathbb{N}. \} \]
Отнормируем собственную функцию ($\tg \varphi = \frac{\mu \beta}{\alpha}$):
\[ 1 = C_{k}^2 \int_{0}^{l} (\cos \sqrt{\lambda_{k}} x + \frac{\alpha}{\beta \mu_{k}} \sin \sqrt{\lambda_{k}} x)^2 = C_{k}^2 (1 + (\frac{\alpha}{\beta \mu_{k}})^2) \int_{0}^{l} \sin^2 (\sqrt{\lambda_{k}}x + \varphi) dx = \]
\[ = C_{k}^2 (1 + (\frac{\alpha}{\beta \mu_{k}})^2) (\frac{x}{2} - \frac{1}{2\sqrt{\lambda_{k}}} \sin (2\sqrt{\lambda_{k}}x + \varphi))_{0}^{l} = C_{k}^2 (1 + (\frac{\alpha}{\beta \mu_{k}})^2) (\frac{l}{2} - \frac{1}{2\sqrt{\lambda_{k}}} (\sin (2\sqrt{\lambda_{k}}l + \varphi) + \sin 2\varphi))\]
Отсюда:
\[ C_{k} = \sqrt{\frac{1}{(1 + (\frac{\alpha}{\beta \sqrt{\lambda_{k}}})^2) (\frac{l}{2} - \frac{1}{2\sqrt{\lambda_{k}}} (\sin (2\sqrt{\lambda_{k}}l + \varphi) + \sin 2\varphi))}} \]

\subsubsection{ $ \alpha > 0, \beta > 0, \gamma = 0, \delta > 0. $}
В этом случае:
\begin{displaymath}
	\begin{cases}
		\alpha C_{1} - \beta \mu C_{2} = 0, \\
		\delta \mu ( -C_{1} \sin \mu l + C_{2} \cos \mu l) = 0
	\end{cases}
\end{displaymath}
Преобразуем:
\begin{displaymath}
	\begin{cases}
		C_{2} = \frac{\alpha}{\beta \mu} C_{1}, \\
		C_{1} ( - \beta \mu\sin \mu l +\alpha \cos \mu l) = 0
	\end{cases}
\end{displaymath}
Интересуемся нетривиальными решениями, поэтому приравняем в нулю скобку во втором уравнении системы, сделав некоторые преобразования:
\[ \sqrt{(\beta \mu)^2 + \alpha^2} (\frac{\alpha}{\sqrt{(\beta \mu)^2 + \alpha^2}} \cos \mu l - \frac{\beta \mu}{\sqrt{(\beta \mu)^2 + \alpha^2}} \sin \mu l) = 0 \]
Свернём получившееся выражение в скобках по формуле косинуса суммы, предварительно обозначив $\tg \Omega = \frac{\beta \mu}{\alpha}$:
\[ \cos (\mu l + \arctg(\frac{\beta \mu}{\alpha})) = 0 \]
\[ \mu l + \arctg(\frac{\beta \mu}{\alpha}) = \frac{\pi}{2} + \pi k, k \in \mathbb{N} \]
Рассуждения аналогичны предыдущему случаю. $C_{1} \ne 0$ и получено нетривиальное решение (или собственная функция), соответствующая собственному числу $\mu ^{*}_{k}$.

\[ C_{1} = const \ne 0, \, C_{2} = \frac{\alpha}{\beta \mu_{k}} C_{1}, \, X(x) = C_{k} (\cos \sqrt{\lambda_{k}} x + \frac{\alpha}{\beta \mu} \sin \sqrt{\lambda_{k}} x), \]
\[ \sqrt{\lambda_{k}} \in \{ \sqrt{\lambda} \, | \, \sqrt{\lambda} l + \arctg \frac{\beta \sqrt{\lambda}}{\alpha} = \pi k, \, k \in \mathbb{N}. \} \]
Отнормируем собственную функцию ($\tg \varphi = \frac{\mu \beta}{\alpha}$):
\[ 1 = C_{k}^2 \int_{0}^{l} (\cos \sqrt{\lambda_{k}} x + \frac{\alpha}{\beta \mu_{k}} \sin \sqrt{\lambda_{k}} x)^2 = C_{k}^2 (1 + (\frac{\alpha}{\beta \mu_{k}})^2) \int_{0}^{l} \sin^2 (\sqrt{\lambda_{k}}x + \varphi) dx = \]
\[ = C_{k}^2 (1 + (\frac{\alpha}{\beta \mu_{k}})^2) (\frac{x}{2} - \frac{1}{2\sqrt{\lambda_{k}}} \sin (2\sqrt{\lambda_{k}}x + \varphi))_{0}^{l} = C_{k}^2 (1 + (\frac{\alpha}{\beta \mu_{k}})^2) (\frac{l}{2} - \frac{1}{2\sqrt{\lambda_{k}}} (\sin (2\sqrt{\lambda_{k}}l + \varphi) + \sin 2\varphi))\]
Отсюда:
\[ C_{k} = \sqrt{\frac{1}{(1 + (\frac{\alpha}{\beta \sqrt{\lambda_{k}}})^2) (\frac{l}{2} - \frac{1}{2\sqrt{\lambda_{k}}} (\sin (2\sqrt{\lambda_{k}}l + \varphi) + \sin 2\varphi))}} \]

\subsubsection{ $ \alpha > 0, \beta > 0, \gamma > 0, \delta > 0. $}
В этом случае:
\begin{displaymath}
	\begin{cases}
		\alpha C_{1} - \beta \mu C_{2} = 0, \\
		\gamma (C_{1} \cos \mu l + C_{2} \sin \mu l) + \delta \mu ( -C_{1} \sin \mu l + C_{2} \cos \mu l) = 0
	\end{cases}
\end{displaymath}
Преобразуем:
\begin{displaymath}
	\begin{cases}
		C_{2} = \frac{\alpha}{\beta \mu} C_{1}, \\
		\gamma (C_{1} \cos \mu l + \frac{\alpha}{\beta \mu} C_{1} \sin \mu l) + \delta \mu ( -C_{1} \sin \mu l + \frac{\alpha}{\beta \mu} C_{1} \cos \mu l) = 0
	\end{cases}
\end{displaymath}
Интересуемся нетривиальными решениями, поэтому сократим на $C_{1}$ и умножим на $\beta \mu$ последнее уравнение в системе:
\[ \gamma (\beta \mu \cos \mu l + \alpha \sin \mu l) + \delta \mu ( - \beta \mu \sin \mu l + \alpha \cos \mu l) = 0 \]
\[ \cos \mu l (\gamma \beta \mu + \alpha \delta \mu) + \sin \mu l (\alpha \gamma - \beta \delta \mu^2) = 0 \]
Обозначим $ A = (\gamma \beta \mu + \alpha \delta \mu) $ и $ B = (\alpha \gamma - \beta \delta \mu^2) $, тогда:
\[ \sqrt{A^2 + B^2} \cos (\mu l - \arctg \frac{B}{A}) = 0 \]
\[ \mu l - \arctg \frac{\alpha \gamma - \beta \delta \mu^2}{\gamma \beta \mu + \alpha \delta \mu} = \frac{\pi}{2} + \pi k, k \in \mathbb{N} \]
\[ \mu l + \arctg \frac{\beta \delta \mu^2 - \alpha \gamma}{\gamma \beta \mu + \alpha \delta \mu} = \frac{\pi}{2} + \pi k, k \in \mathbb{N} \]
Рассуждения аналогичны предыдущему случаю. $C_{1} \ne 0$ и получено нетривиальное решение (или собственная функция), соответствующая собственному числу $\mu ^{*}_{k}$. 
\[ C_{1} = const \ne 0, \, C_{2} = \frac{\alpha}{\beta \mu_{k}} C_{1}, \, X(x) = C_{k} (\cos \sqrt{\lambda_{k}} x + \frac{\alpha}{\beta \mu} \sin \sqrt{\lambda_{k}} x), \]
\[ \sqrt{\lambda_{k}} \in \{ \sqrt{\lambda} \, | \, \mu l + \arctg \frac{\beta \delta \lambda - \alpha \gamma}{\gamma \beta \sqrt{\lambda} + \alpha \delta \sqrt{\lambda}} = \frac{\pi}{2} + \pi k, k \in \mathbb{N}. \} \]
Отнормируем собственную функцию ($\tg \varphi = \frac{\mu \beta}{\alpha}$):
\[ 1 = C_{k}^2 \int_{0}^{l} (\cos \sqrt{\lambda_{k}} x + \frac{\alpha}{\beta \mu_{k}} \sin \sqrt{\lambda_{k}} x)^2 = C_{k}^2 (1 + (\frac{\alpha}{\beta \mu_{k}})^2) \int_{0}^{l} \sin^2 (\sqrt{\lambda_{k}}x + \varphi) dx = \]
\[ = C_{k}^2 (1 + (\frac{\alpha}{\beta \mu_{k}})^2) (\frac{x}{2} - \frac{1}{2\sqrt{\lambda_{k}}} \sin (2\sqrt{\lambda_{k}}x + \varphi))_{0}^{l} = C_{k}^2 (1 + (\frac{\alpha}{\beta \mu_{k}})^2) (\frac{l}{2} - \frac{1}{2\sqrt{\lambda_{k}}} (\sin (2\sqrt{\lambda_{k}}l + \varphi) + \sin 2\varphi))\]
Отсюда:
\[ C_{k} = \sqrt{\frac{1}{(1 + (\frac{\alpha}{\beta \sqrt{\lambda_{k}}})^2) (\frac{l}{2} - \frac{1}{2\sqrt{\lambda_{k}}} (\sin (2\sqrt{\lambda_{k}}l + \varphi) + \sin 2\varphi))}} \]

\section{Численное построение графиков собственных функций}
Для последнего случая при $\lambda > 0$ были найдены численно пять собственных значений, построены графики собственных функций, отвечающих этим собственным значениям, а также был построен график изменения амплитуды колебания с ростом значения собственного числа.
\newline
Численно найденные собственные значения: $\lambda_{1} = 1.52170$, $\lambda_{2} = 2.30388$, $\lambda_{3} = 3.38247$, $\lambda_{4} = 4.55452$, $\lambda_{5} = 8.22286$.
\newline
Построенные графики представлены ниже.

\begin{figure}[H]
	\center{\includegraphics[scale=0.4]{UMF_PIC}}
	\caption{Численно построенные графики}
\end{figure}

На графиках собственных функций жёлтая линия соответствует собственному значению $\lambda_{1} = 1.52170$, зелёная - 
$\lambda_{2} = 2.30388$, фиолетовая - $\lambda_{3} = 3.38247$, красная -  $\lambda_{4} = 4.55452$ и синяя - $\lambda_{5} = 8.22286$.
Как видно из последнего графика, показывающего зависимость амплитуды колебаний от собственного значения, амплитуда колебаний при большом собственном значении оказывается примерно равным $\sqrt{\frac{2}{l}}$. Это же подтверждается и аналитически

\[ \lim\limits_{\lambda \to +\infty} \sqrt{\frac{1}{(1 + (\frac{\alpha}{\beta \sqrt{\lambda}})^2) (\frac{l}{2} - \frac{1}{2\sqrt{\lambda}} (\sin (2\sqrt{\lambda}l + \varphi) + \sin 2\varphi))}} =  \sqrt{\frac{2}{l}}. \]

\section{Основные свойства собственных чисел и собственных функций}
\begin{enumerate}
	\item Существует счётное число собственных значений $\lambda_1 < \lambda_2 < ... < \lambda_n < ... $, которым соответсвуют нетривиальные решения задачи - собственные функции $X_1 (x), X_2 (x), ..., X_n (x), ... $
	\item Собственные функции $X_n (x), X_m (x), n \neq m$ ортогональны между собой с весом $\rho (x)$ на отрезке $0 \leq x \leq l$.
\[ \int_{0}^{l} X_{n} (x) X_{m} (x) \rho (x) dx = 0, \quad (n \neq m). \]
	\item Если $X_n (x)$ является собственной функцией при собственном значении $\lambda_n$, то функция $A_n X_n (x)$ ($A_n$ - произвольная постоянная) также является собственной функцией для того же значения.
\end{enumerate}

Чтобы исключить неопределённость в выборе множителся, можно подчинить собственные функции требованию нормировки:
\[ ||X_n (x) ||^2 = \int_{0}^{l}  X^2_n (x) \rho (x) dx = 1. \] 
Тогда такие собственные функции образуют ортогональную и нормированную систему:
\begin{equation*}
	 \int_{0}^{l}  X_{n} (x) X_{m} (x) \rho (x) dx =  \begin{cases}
	0, n \neq m, \\
	1, n = m.
	\end{cases}
\end{equation*}

\section{Определения замкнтуности и полноты ортонормированной системы}
\textbf{Определение.} Ортонормированная система ${X_n (x)}$  называется \textit{замкнутой}, если каждая функция $f(x)$ может быть разложена в сходящийся в среднем ряд Фурье по функциям данной системы.
\[ \forall \varepsilon > 0 \quad \forall f \quad \exists f_1 f_2 ... f_n \quad | \quad ||\sum_{i = 1}^n f_i X_i (x) - f(x) || < \varepsilon  \]
\textbf{Определение.} Ортонормированная система ${X_n (x)}$  называется \textit{полной}, если не существует такой нетривиальной функции $f(x)$, что $\int_{0}^{l} f(x) X_n (x) \rho (x) dx = 0 \quad \forall n$.
\section{Определения поточечной и равномерной сходимости рядов Фурье. Теорема Стеклова о равномерной сходимости}
\textbf{Определение.} Ряд Фурье сходится поточечно , если
\[ \forall x \in [0, l] \quad \forall \varepsilon > 0 \quad \exists N_\varepsilon \quad \forall n > N \quad || \sum_{i = 1}^{n} f_i X_i (x) - f(x) || < \varepsilon\]
\textbf{Определение.} Ряд Фурье сходится равномерно, если
\[ \forall \varepsilon > 0 \quad \exists N_\varepsilon \quad \forall n > N \quad \forall x \in [0, l] \quad || \sum_{i = 1}^{n} f_i X_i (x) - f(x) || < \varepsilon\]
\textbf{Теорема В.А. Стеклова.} Произвольная, дважды непрерывно дифференцируемая функция $f(x)$, удовляетворяющая граничным условиям $f(0) = f(l) = 0$, разлагается в равномерно и абсолютно сходящийся ряд по собственным функциям $X_{n} (x)$:
\[ f(x) = \sum_{n = 1}^{\infty} f_n X_n (x) \quad f_n = \frac{1}{||X_n (x)||^2} \int_{0}^{l} f(x) X_n (x) \rho (x) dx \quad ||X_n (x) ||^2 = \int_{0}^{l}  X^2_n (x) \rho (x) dx \]

\chapter{Специальные функции}

\section{Функции Бесселя}
\subsection{Найти фундаментальную систему решений уравнения Бесселя $\nu-$го порядка}

Уравнение имеет вид:
\[ y'' + \frac{y'}{x} + (1 - \frac{\nu^2}{x^2}) y = 0, \]
где $\nu \in \mathbb{R}$ или $\nu \in \mathbb{C} \quad | \quad Re\nu < 0$.

Решение будем искать в виде:
\[ y = x^{\sigma} \sum_{n = 0}^{+\infty} a_{n} x^{n}, \quad a_{0} \neq 0, \]
поскольку уравнение имеет особенность в точке $x = 0$, $\sigma$ называется характеристическим показателем.

Подставим ряд в уравнение и соберём слагаемые при $x^{n + \sigma - 2}$:
\[ \sum_{n = 0}^{+\infty} (n + \sigma)(n + \sigma - 1)a_{n} x^{n + \sigma - 2} + (n+ \sigma)a_{n} x^{n + \sigma - 2} + a_{n} x^{n + \sigma} - \nu^2 a_{n} x^{n + \sigma - 2} = 0 \]
\[ \sum_{n = 0}^{+\infty} a_{n} x^{n + \sigma - 2} ((n + \sigma)^2 - \nu^2 + x^2) = \sum_{n = 0}^{+\infty} a_{n} x^{n + \sigma - 2} ((n + \sigma)^2 - \nu^2) + \sum_{n = 0}^{+\infty} a_{n} x^{n + \sigma} = 0\]

Приравняем к нулю коэффициенты при степенях $\sigma - 2$, $\sigma - 1$, ..., $n + \sigma - 2$:
\begin{equation*}
	\begin{cases}
		a_{0} (\sigma^2 - \nu^2) = 0 \\
		a_{1}((1 + \sigma)^2 - \nu^2) = 0 \\
		a_{2}((2 + \sigma)^2 - \nu^2) + a_{0} = 0 \\
		a_{3}((3 + \sigma)^2 - \nu^2) + a_{1} = 0 \\
		... \\
		a_{n}((n + \sigma)^2 - \nu^2) + a_{n - 2} = 0
	\end{cases}
\end{equation*}

Поскольку $a_{0} \neq 0$ по условию, то $\sigma^2 - \nu^2 = 0$ и $\sigma = \pm \nu$. Для второго уравнения системы с учётом предыдущей выкладки имеем:
\[ a_{1} (1 + 2\sigma + \sigma^2 - \nu^2) = a_{1} (1 + 2\sigma) = 0 \Rightarrow a_{1} = 0. \]

$\forall n > 1$ получим рекуррентную формулу для коэффициентов $a_{n}$:

\[ a_{n} (n + \sigma - \nu)(n + \sigma + \nu) + a_{n - 2} = 0 \]
\[ a_{n} = - \frac{a_{n - 2}}{(n + \sigma - \nu)(n + \sigma + \nu)} \]

Поскольку ранее было показано, что $a_{1} = 0$, то из рекуррентной формулы можно увидеть, что все $a_{2l + 1} = 0, \quad l > 0$.

\end{document}