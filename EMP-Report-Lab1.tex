\documentclass[12pt, a4paper]{article}

\usepackage[T2A]{fontenc}
\usepackage[utf8]{inputenc}
\usepackage[english,russian]{babel}
\usepackage[left = 1 cm, right = 1 cm, top = 2cm, bottom = 2 cm, bindingoffset = 0 cm]{geometry}
\usepackage{amsmath,amsfonts,amssymb,amsthm,mathtools}
\usepackage{wasysym}

\usepackage{graphicx}
\graphicspath{{pictures/}}
\DeclareGraphicsExtensions{.pdf,.png,.jpg}

\usepackage{alltt}

\begin{document}
\begin{titlepage}
\newpage

\begin{center}
Министерство образования и науки Российской Федерации \\
Федеральное государственное автономное образовательное
учреждение высшего образования \\
Национальный исследовательский Нижегородский государственный
университет им. Н.И. Лобачевского \\
Институт информационных технологий, математики и механики \\
\end{center}

\vspace{12em}

\begin{center}
\textsc{\textbf{Отчёт по лабораторной работе:}}\\
\textsc{\textbf{"Задача Штурма-Лиувилля"}}
\end{center}

\vspace{14em}



\newbox{\lbox}
\savebox{\lbox}{\hbox{Петров Павел, Михайлова Екатерина}}
\newlength{\maxl}
\setlength{\maxl}{\wd\lbox}
\hfill\parbox{11cm}{
\hspace*{5cm}\hspace*{-5cm}Студенты:\hfill\hbox to\maxl{Петров Павел, Михайлова Екатерина\hfill}\\
\\
\hspace*{5cm}\hspace*{-5cm}Группа:\hfill\hbox to\maxl{381803-1}\\
}


\vspace{\fill}

\begin{center}
Нижний Новгород \\2020
\end{center}

\end{titlepage}       


\section{Задача Штурма-Лиувилля}

Задача Штурма-Лиувилля является простейшей задачей о поиске ортонормированной системы: найти те значения параметра $\lambda$, при которых существует нетривиальные решения задачи:
\[ X'' + \lambda X = 0 \]
\[ \alpha X(0) - \beta X'(0) = 0, \quad \alpha \geq 0, \beta \geq 0, \alpha + \beta > 0, \]
\[ \gamma X(l) + \delta X'(l) = 0, \quad \gamma \geq 0, \delta \geq 0, \gamma + \delta > 0, \]
а также найти эти решения. Такие значения параметра $\lambda$ называются \textbf{собственными значениями}, а соответствующие им нетривиальные решения - \textbf{собственными функциями}.


Из анализа ограничений на параметры были получены девять случаев значений параметров:
\begin{enumerate}
	\item $ \alpha > 0, \beta = 0, \gamma > 0, \delta = 0. $
	\item $ \alpha > 0, \beta = 0, \gamma = 0, \delta > 0. $
	\item $ \alpha = 0, \beta > 0, \gamma > 0, \delta = 0. $
	\item $ \alpha = 0, \beta > 0, \gamma = 0, \delta > 0. $
	\item $ \alpha > 0, \beta = 0, \gamma > 0, \delta > 0. $
	\item $ \alpha = 0, \beta > 0, \gamma > 0, \delta > 0. $
	\item $ \alpha > 0, \beta > 0, \gamma > 0, \delta = 0. $
	\item $ \alpha > 0, \beta > 0, \gamma = 0, \delta > 0. $
	\item $ \alpha > 0, \beta > 0, \gamma > 0, \delta > 0. $
\end{enumerate}


\section{Для всех возможных девяти случаев найти собственные числа и собственный функции задачи Штурма-Лиувилля (собственные функции отнормировать!)}
Рассмотрим разные значения $\lambda$ в задаче Штурма-Лиувилля.
\subsection{$\lambda = 0$}
В этом случае уравнение имеет вид:
\[ X'' = 0, \]
а его решение:
\[ X(x) = C_{1} x + C_{2}, \]
где $C_{1}$ и $C_{2}$ - произвольные постоянные. Подставим это решение в граничные условия и получим, что:
\[\alpha C_{2} - \beta C_{1} = 0, \quad \gamma (C_{1} l + C_{2}) + \delta C_{1} = 0. \]

\subsubsection{ $ \alpha > 0, \beta = 0, \gamma > 0, \delta = 0. $}
В этом случае:
\[\alpha C_{2} = 0, \quad \gamma (C_{1} l + C_{2}) = 0, \]
\[ C_{2} = 0, \quad  \gamma C_{1} l = 0. \]
Так как по условию $\gamma$ и $l$ отличны от нуля, то равенство последнего уравнения нулю возможно только в одном случае, когда $C_{1} = 0$. Как итог:
\[ C_{1} = 0, \, C_{2} = 0, \, X(x) = 0. \]
Получили тривиальное решение, но оно нас не интересует.

\subsubsection{ $ \alpha > 0, \beta = 0, \gamma = 0, \delta > 0. $}
В этом случае:
\[\alpha C_{2} = 0, \quad \delta C_{1} = 0. \]
Исходя из условий, получаем:
\[ C_{1} = 0, \, C_{2} = 0, \, X(x) = 0. \]
Получили тривиальное решение, но оно нас не интересует.

\subsubsection{ $ \alpha = 0, \beta > 0, \gamma > 0, \delta = 0. $}
В этом случае:
\[ - \beta C_{1} = 0, \quad \gamma (C_{1} l + C_{2}) = 0, \]
\[ C_{1} = 0, \quad \gamma C_{2} = 0. \]
Так как по условию $\gamma$ отлична от нуля, то равенство последнего уравнения нулю возможно только в одном случае, когда $C_{2} = 0$. Как итог:
\[ C_{1} = 0, \, C_{2} = 0, \, X(x) = 0. \]
Получили тривиальное решение, но оно нас не интересует.

\subsubsection{ $ \alpha = 0, \beta > 0, \gamma = 0, \delta > 0. $}
В этом случае:
\[-\beta C_{1} = 0, \quad \delta C_{1} = 0. \]
Исходя из условий, получаем, что $C_{1} = 0$, а на $C_{2}$ ограничений нет. Значит, чтобы получить нетривиальное решение, надо взять $C_{2}$ произвольной константой, отличной от нуля. Итог:
\[ C_{1} = 0, \, C_{2} = const \neq 0, \, X(x) = C_{2}. \]
Получили нетривиальное решение.

\subsubsection{ $ \alpha > 0, \beta = 0, \gamma > 0, \delta > 0. $}
В этом случае:
\[ \alpha C_{2} = 0, \quad \gamma (C_{1} l + C_{2}) + \delta C_{1} = 0, \]
\[ C_{2} = 0, \quad (\gamma l + \delta) C_{1} = 0. \]
Так как по условию $\gamma$, $l$ и $\delta$ строго больше нуля, то равенство последнего уравнения нулю возможно только в одном случае, когда $C_{1} = 0$. Как итог:
\[ C_{1} = 0, \, C_{2} = 0, \, X(x) = 0. \]
Получили тривиальное решение, но оно нас не интересует.

\subsubsection{ $ \alpha = 0, \beta > 0, \gamma > 0, \delta > 0. $}
В этом случае:
\[ - \beta C_{1} = 0, \quad \gamma (C_{1} l + C_{2}) + \delta C_{1} = 0, \]
\[ C_{1} = 0, \quad \gamma C_{2} = 0. \]
Так как по условию $\gamma$ отлична от нуля, то равенство последнего уравнения нулю возможно только в одном случае, когда $C_{2} = 0$. Как итог:
\[ C_{1} = 0, \, C_{2} = 0, \, X(x) = 0. \]
Получили тривиальное решение, но оно нас не интересует.

\subsubsection{ $ \alpha > 0, \beta > 0, \gamma > 0, \delta = 0. $}
В этом случае:
\[ \alpha C_{2} - \beta C_{1} = 0, \quad \gamma (C_{1} l + C_{2}) = 0, \]
\[ C_{2} = \frac{\beta}{\alpha} C_{1}, \quad (\gamma l + \frac{\gamma \beta}{\alpha}) C_{1} = 0. \]
Так как по условию все представленные параметры строго больше нуля, то равенство последнего уравнения нулю возможно только в одном случае, когда $C_{1} = 0$, следовательно и $C_{2} = 0$. Как итог:
\[ C_{1} = 0, \, C_{2} = 0, \, X(x) = 0. \]
Получили тривиальное решение, но оно нас не интересует.

\subsubsection{ $ \alpha > 0, \beta > 0, \gamma = 0, \delta > 0. $}
В этом случае:
\[ \alpha C_{2} - \beta C_{1} = 0, \quad \delta C_{1} = 0, \]
\[ \alpha C_{2} = 0, \quad C_{1} = 0. \]
Так как по условию $\alpha$ строго больше нуля, то равенство первого уравнения нулю возможно только в одном случае, когда $C_{2} = 0$. Как итог:
\[ C_{1} = 0, \, C_{2} = 0, \, X(x) = 0. \]
Получили тривиальное решение, но оно нас не интересует.

\subsubsection{ $ \alpha > 0, \beta > 0, \gamma > 0, \delta > 0. $}
В этом случае:
\[ \alpha C_{2} - \beta C_{1} = 0, \quad \gamma (C_{1} l + C_{2}) + \delta C_{1} = 0, \]
\[ C_{2} = \frac{\beta}{\alpha} C_{1}, \quad (\gamma l + \delta + \frac{\gamma \beta}{\alpha}) C_{1} = 0. \]
Так как по условию все представленные параметры строго больше нуля, то равенство последнего уравнения нулю возможно только в одном случае, когда $C_{1} = 0$, следовательно и $C_{2} = 0$. Как итог:
\[ C_{1} = 0, \, C_{2} = 0, \, X(x) = 0. \]
Получили тривиальное решение, но оно нас не интересует.

\subsection{$\lambda < 0$}
В этом случае решение уравнения имеет вид:
\[ X(x) = C_{1} e^{-\sqrt{-\lambda}x} + C_{2} e^{\sqrt{-\lambda}x}. \]
Заменим $\sqrt{-\lambda}$ на $\mu$ и подставим решение в граничные условия. Получим:
\begin{equation}
	\begin{cases}
		\alpha (C_{1} + C_{2}) - \beta \mu (C_{2} - C_{1}) = 0, \\
		\gamma (C_{1} e^{-\mu l} + C_{2} e^{\mu l}) + \delta \mu (C_{2} e^{\mu l} - C_{1} e^{-\mu l}) = 0
	\end{cases}
\end{equation}

\subsubsection{ $ \alpha > 0, \beta = 0, \gamma > 0, \delta = 0. $}
В этом случае:
\begin{equation}
	\begin{cases}
		\alpha (C_{1} + C_{2}) = 0, \\
		\gamma (C_{1} e^{-\mu l} + C_{2} e^{\mu l}) = 0
	\end{cases}
\end{equation}

\begin{equation}
	\begin{cases}
		C_{1} = - C_{2}, \\
		\gamma C_{1}(e^{-\mu l} - e^{\mu l}) = 0
	\end{cases}
\end{equation}
Интересуемся нетривиальными решениями, поэтому приравняем $e^{-\mu l} - e^{\mu l}$ к нулю. Но здесь равенство нулю возможно только в случае равенства нулю $\mu l$, что невозможно, исходя из условий. Поэтому возможно только $C_{1} = 0$ и, следовательно, $C_{2} = 0$. Таким образом получаем тривиальное решение, что нам не подходит.

\[ C_{1} = 0, \, C_{2} = 0, \, X(x) = 0. \]

\subsubsection{ $ \alpha > 0, \beta = 0, \gamma = 0, \delta > 0. $}
В этом случае:
\begin{equation}
	\begin{cases}
		\alpha (C_{1} + C_{2}) = 0, \\
		\delta \mu (C_{2} e^{\mu l} - C_{1} e^{-\mu l}) = 0
	\end{cases}
\end{equation}

\begin{equation}
	\begin{cases}
		C_{1} = - C_{2}, \\
		-\delta \mu C_{1} (e^{\mu l} + e^{-\mu l}) = 0
	\end{cases}
\end{equation}
Второе уравнения равно нулю только в одном случае, когда $C_{1} = 0$, так как по условию параметры строго больше нуля, а $e^{\mu l} + e^{-\mu l}$ никогда не может быть равно нулю, так как представляет собой сумму положительных функций. Следовательно, $C_{2} = 0$, и получаем тривиальное решение, что нам не подходит.

\[ C_{1} = 0, \, C_{2} = 0, \, X(x) = 0. \]

\subsubsection{ $ \alpha = 0, \beta > 0, \gamma > 0, \delta = 0. $}
В этом случае:
\begin{equation}
	\begin{cases}
		- \beta \mu (C_{2} - C_{1}) = 0, \\
		\gamma (C_{1} e^{-\mu l} + C_{2} e^{\mu l}) = 0
	\end{cases}
\end{equation}

\begin{equation}
	\begin{cases}
		C_{2} = C_{1}, \\
		\gamma C_{1} (e^{-\mu l} + e^{\mu l}) = 0
	\end{cases}
\end{equation}
Второе уравнения равно нулю только в одном случае, когда $C_{1} = 0$, так как по условию параметр строго больше нуля, а $e^{\mu l} + e^{-\mu l}$ никогда не может быть равно нулю, так как представляет собой сумму положительных функций. Следовательно, $C_{2} = 0$, и получаем тривиальное решение, что нам не подходит.

\[ C_{1} = 0, \, C_{2} = 0, \, X(x) = 0. \]

\subsubsection{ $ \alpha = 0, \beta > 0, \gamma = 0, \delta > 0. $}
В этом случае:
\begin{equation}
	\begin{cases}
		- \beta \mu (C_{2} - C_{1}) = 0, \\
		\delta \mu (C_{2} e^{\mu l} - C_{1} e^{-\mu l}) = 0
	\end{cases}
\end{equation}

\begin{equation}
	\begin{cases}
		C_{2} = C_{1}, \\
		\delta \mu C_{1} (e^{\mu l} - e^{-\mu l}) = 0
	\end{cases}
\end{equation}
Интересуемся нетривиальными решениями, поэтому приравняем $e^{\mu l} - e^{-\mu l}$ к нулю. Но здесь равенство нулю возможно только в случае равенства нулю $\mu l$, что невозможно, исходя из условий. Поэтому возможно только $C_{1} = 0$ и, следовательно, $C_{2} = 0$. Таким образом получаем тривиальное решение, что нам не подходит.

\[ C_{1} = 0, \, C_{2} = 0, \, X(x) = 0. \]

\subsubsection{ $ \alpha > 0, \beta = 0, \gamma > 0, \delta > 0. $}
В этом случае:
\begin{equation}
	\begin{cases}
		\alpha (C_{1} + C_{2}) = 0, \\
		\gamma (C_{1} e^{-\mu l} + C_{2} e^{\mu l}) + \delta \mu (C_{2} e^{\mu l} - C_{1} e^{-\mu l}) = 0
	\end{cases}
\end{equation}

\begin{equation}
	\begin{cases}
		C_{1} = -C_{2}, \\
		C_{1} (\gamma (e^{-\mu l} - e^{\mu l}) - \delta \mu (e^{\mu l} + e^{-\mu l})) = 0
	\end{cases}
\end{equation}
Интересуемся нетривиальными решениями, поэтому во втором уравнении занулим скобку:
\[ \gamma (e^{-\mu l} - e^{\mu l}) - \delta \mu (e^{\mu l} + e^{-\mu l}) = 0, \]
затем умножим на $ e^{\mu l}$ и соберём слагаемые с экспонентами и без:
\[ (\gamma - \delta \mu) -  (\gamma + \delta \mu) e^{2 \mu l} = 0. \]
Получаем уравнение, которое надо решить относительно $\mu$:
\[ e^{2 \mu l} = \frac{\gamma - \delta \mu}{\gamma + \delta \mu}. \]
Слева в уравнении, очевидно, стоит монотонно возрастающая функция. Справа имеем гиперболу с вертикальной асимптотой $\mu = -\frac{\gamma}{\delta} < 0$. Но $\mu = \sqrt{-\lambda}$, поэтому рассматриваем только положительные $\mu$. Производная функции справа по $\mu$ равна: $\frac{-2\delta \gamma}{(\delta \mu + \gamma)^2} < 0$, значит гипербола убывает на всей своей области определения. При $\mu = 0$ функции совпадают, затем расходятся, поэтому при положительных $\mu$ обращение скобки в нуль невозможно. Значит возможно только $C_{1} = 0$ и, следовательно, $C_{2} = 0$, и решение тривиально.

\[ C_{1} = 0, \, C_{2} = 0, \, X(x) = 0. \]

\subsubsection{ $ \alpha = 0, \beta > 0, \gamma > 0, \delta > 0. $}
В этом случае: 
\begin{equation}
	\begin{cases}
		- \beta \mu (C_{2} - C_{1}) = 0, \\
		\gamma (C_{1} e^{-\mu l} + C_{2} e^{\mu l}) + \delta \mu (C_{2} e^{\mu l} - C_{1} e^{-\mu l}) = 0
	\end{cases}
\end{equation}

\begin{equation}
	\begin{cases}
		C_{2} = C_{1}, \\
		C_{1} (\gamma (e^{-\mu l} + e^{\mu l}) + \delta \mu (e^{\mu l} - e^{-\mu l})) = 0
	\end{cases}
\end{equation}
Интересуемся нетривиальными решениями, поэтому во втором уравнении занулим скобку:
\[ \gamma (e^{-\mu l} + e^{\mu l}) + \delta \mu (e^{\mu l} - e^{-\mu l}) = 0, \]
затем умножим на $ e^{\mu l}$ и соберём слагаемые с экспонентами и без:
\[ (\gamma - \delta \mu) + (\gamma + \delta \mu) e^{2 \mu l} = 0. \]
Получаем уравнение, которое надо решить относительно $\mu$:
\[ e^{2 \mu l} = \frac{\delta \mu - \gamma}{\delta \mu + \gamma}. \]
Слева в уравнении, очевидно, стоит монотонно возрастающая функция. Справа имеем гиперболу с вертикальной асимптотой $\mu = -\frac{\gamma}{\delta} < 0$. Но $\mu = \sqrt{-\lambda}$, поэтому рассматриваем только положительные $\mu$. Производная функции справа по $\mu$ равна: $\frac{2\delta \gamma}{(\delta \mu + \gamma)^2} < 0$, значит гипербола возрастает на всей своей области определения. При $\mu = 0$ функция слева равна 1, справа равна -1, при стремлении $\mu$ к бесконечности, гипербола стремится к 1, пока экспонента в это время уходит на бесконечность, поэтому нет такого значения $\mu$, при котором скобка обращалась бы в нуль. Значит возможно только $C_{1} = 0$ и, следовательно, $C_{2} = 0$, и решение тривиально.

\[ C_{1} = 0, \, C_{2} = 0, \, X(x) = 0. \]

\subsubsection{ $ \alpha > 0, \beta > 0, \gamma > 0, \delta = 0. $}
\subsubsection{ $ \alpha > 0, \beta > 0, \gamma = 0, \delta > 0. $}
В этом случае:
\begin{equation}
	\begin{cases}
		\alpha (C_{1} + C_{2}) - \beta \mu (C_{2} - C_{1}) = 0, \\
		\delta \mu (C_{2} e^{\mu l} - C_{1} e^{-\mu l}) = 0
	\end{cases}
\end{equation}
Выразим $C_{2}$ через $C_{1}$ в первом уравнении:
\begin{equation}
	\begin{cases}
		C_{2} = -\frac{\beta \mu + \alpha}{\beta \mu - \alpha} C_{1}, \\
		- \delta \mu C_{1} (\frac{\beta \mu + \alpha}{\beta \mu - \alpha} e^{\mu l} + e^{-\mu l}) = 0
	\end{cases}
\end{equation}
Домножим последнее уравнение на $e^{\mu l}$ и $\frac{\beta \mu - \alpha}{\beta \mu + \alpha}$ и занулим скобку:
\[ e^{2\mu l} = -\frac{\beta \mu - \alpha}{\beta \mu + \alpha} \]
Слева в уравнении, очевидно, стоит монотонно возрастающая функция. Справа имеем гиперболу с вертикальной асимптотой $\mu = -\frac{\alpha}{\beta} < 0$. Но $\mu = \sqrt{-\lambda}$, поэтому рассматриваем только положительные $\mu$. Производная функции справа по $\mu$ равна: $\frac{-2\alpha \beta}{(\beta \mu + \alpha)^2} < 0$, значит гипербола убывает на всей своей области определения. При $\mu = 0$ функции совпадают, затем расходятся, поэтому при положительных $\mu$ обращение скобки в нуль невозможно. Значит возможно только $C_{1} = 0$ и, следовательно, $C_{2} = 0$, и решение тривиально.

\[ C_{1} = 0, \, C_{2} = 0, \, X(x) = 0. \]

\subsubsection{ $ \alpha > 0, \beta > 0, \gamma > 0, \delta > 0. $}
В этом случае:
\begin{equation}
	\begin{cases}
		\alpha (C_{1} + C_{2}) - \beta \mu (C_{2} - C_{1}) = 0, \\
		\gamma (C_{1} e^{-\mu l} + C_{2} e^{\mu l}) + \delta \mu (C_{2} e^{\mu l} - C_{1} e^{-\mu l}) = 0
	\end{cases}
\end{equation}
Выразим $C_{2}$ через $C_{1}$ в первом уравнении:
\begin{equation}
	\begin{cases}
		C_{2} = -\frac{\beta \mu + \alpha}{\beta \mu - \alpha} C_{1}, \\
		\gamma (C_{1} e^{-\mu l} -\frac{\beta \mu + \alpha}{\beta \mu - \alpha} C_{1} e^{\mu l}) + \delta \mu (-\frac{\beta \mu + \alpha}{\beta \mu - \alpha} C_{1} e^{\mu l} - C_{1} e^{-\mu l}) = 0
	\end{cases}
\end{equation}
Умножим последнее уравнение на $e^{\mu l} и вынесем C_{1}$:
\begin{equation}
	\begin{cases}
		C_{2} = -\frac{\beta \mu + \alpha}{\beta \mu - \alpha} C_{1}, \\
		\gamma C_{1}(1 -\frac{\beta \mu + \alpha}{\beta \mu - \alpha} e^{2 \mu l}) + \delta \mu C_{1} (-\frac{\beta \mu + \alpha}{\beta \mu - \alpha} e^{2\mu l} - 1) = 0
	\end{cases}
\end{equation}
Преобразуем второй уравнение:
\[ C_{1} ( -\frac{\beta \mu + \alpha}{\beta \mu - \alpha} (\gamma + \delta \mu ) e^{2 \mu l} - (\delta \mu - \gamma)) = 0 \]
Приравняем скобку к нулю и получим уравнение, которое нужно решить относительно $\mu$:
\[ e^{2\mu l} = - \frac{\beta \mu - \alpha}{\beta \mu + \alpha} \frac{\delta \mu - \gamma}{\delta \mu + \gamma}\]
При $\mu = 0$ экспонента равна 1, а функция справа -1, на бесконечности экспонента устремляется к бесконечности, монотонно возрастая, а функция справа стремится к -1, причём скорость роста экспоненты выше, чем скорость функции справа, поэтому пересечения графиков быть не может. Следовательно, скобка не может обратиться к нуль, а значит равенство нулю второго уравнения системы возможно только лишь в случае $C_{1} = 0$. Из этого следует, что $C_{2} = 0$ и решение тривиально.

\[ C_{1} = 0, \, C_{2} = 0, \, X(x) = 0. \]

\subsection{$\lambda > 0$}

В этом случае решение уравнения имеет вид:
\[ X(x) = C_{1} \cos \sqrt{\lambda} x + C_{2} \sin \sqrt{\lambda}x. \]
Заменим $\sqrt{\lambda}$ на $\mu$ и подставим решение в граничные условия. Получим:
\begin{equation}
	\begin{cases}
		\alpha C_{1} - \beta \mu C_{2} = 0, \\
		\gamma (C_{1} \cos \mu l + C_{2} \sin \mu l) + \delta \mu ( -C_{1} \sin \mu l + C_{2} \cos \mu l) = 0
	\end{cases}
\end{equation}

\subsubsection{ $ \alpha > 0, \beta = 0, \gamma > 0, \delta = 0. $}
\subsubsection{ $ \alpha > 0, \beta = 0, \gamma = 0, \delta > 0. $}
\subsubsection{ $ \alpha = 0, \beta > 0, \gamma > 0, \delta = 0. $}
\subsubsection{ $ \alpha = 0, \beta > 0, \gamma = 0, \delta > 0. $}
\subsubsection{ $ \alpha > 0, \beta = 0, \gamma > 0, \delta > 0. $}
\subsubsection{ $ \alpha = 0, \beta > 0, \gamma > 0, \delta > 0. $}
В этом случае:
\begin{equation}
	\begin{cases}
		C_{2} = 0, \\
		C_{1} (\gamma \cos \mu l - \delta \mu \sin \mu l) = 0
	\end{cases}
\end{equation}
Ищем нетривиальные решения, поэтому занулим во втором уравнении скобку и сделаем некоторое преобразование:
\[ \sqrt{\gamma ^2 + (\delta \mu)^2 }(\frac{\gamma}{\sqrt{\gamma ^2 + (\delta \mu)^2 }} \cos \mu l - \frac{\delta \mu}{\sqrt{\gamma ^2 + (\delta \mu)^2 }}\sin \mu l) = 0 \]
Обозначим $\tg \Omega = \frac{\delta \mu}{\gamma}$ и свернём получившееся выражение в скобке по формуле косинуса суммы:
\[ \sqrt{\gamma ^2 + (\delta \mu)^2 } cos(\mu l + \arctg  \frac{\delta \mu}{\gamma}) = 0 \]
Из условий на параметры и на собственное число получим, что:
\[ \mu l + \arctg  \frac{\delta \mu}{\gamma} = \frac{\pi}{2} + \pi k, \, k \in \mathbb{N}. \]
Слева получили, очевидно, монотонно возрастающую функцию, справа уравнение горизонтальной прямой. Понятно, что, в силу свойств функций, обязательно найдётся при каждом $k$ такое $\mu ^{*}_{k}$, которое будет удовлетворять полученному уравнению. Значит $C_{1} \ne 0$ и получено нетривиальное решение (или собственная функция), соответствующая собственному числу $\mu ^{*}_{k}$.

\[ C_{1} = const \ne 0, \, C_{2} = 0, \, X(x) = C_{1} \cos \sqrt{\lambda_{k}} x, \, \sqrt{\lambda_{k}} \in \{\sqrt{\lambda} \, | \sqrt{\lambda} l + \arctg  \frac{\delta \sqrt{\lambda}}{\gamma} = \frac{\pi}{2} + \pi k, \, k \in \mathbb{N}\}. \]
Отнормируем собственную функцию:
\[ 1 = (C_{k} \cos \sqrt{\lambda_{k}} x, C_{k} \cos \sqrt{\lambda_{k}} x) = C_{k}^2 \int_{0}^{l}  \cos^2 \sqrt{\lambda_{k}} xdx = C_{k}^2 (\frac{x}{2} + \frac{1}{2\lambda_{k}} \sin 2\lambda_{k} x)_0^l =   C_{k}^2 (\frac{l}{2} + \frac{1}{2\lambda_{k}} \sin 2\lambda_{k} l) \]
Отсюда:
\[ C_{k} = \sqrt{\frac{1}{\frac{l}{2} + \frac{1}{2\lambda_{k}} \sin 2\lambda_{k} l}}. \]

\subsubsection{ $ \alpha > 0, \beta > 0, \gamma > 0, \delta = 0. $}
\subsubsection{ $ \alpha > 0, \beta > 0, \gamma = 0, \delta > 0. $}
\subsubsection{ $ \alpha > 0, \beta > 0, \gamma > 0, \delta > 0. $}

\end{document}